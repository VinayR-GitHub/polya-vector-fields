\documentclass[12pt]{article}
\title{The Mathematics of Pólya Vector Fields}
\author{Vinay R}
\usepackage{amssymb}
\begin{document}
\maketitle

Let $f(z)$ be a function of $z \in \textup{dom}(f)$ with $\textup{dom}(f) \subset \mathbb{C}$.
Let there exist a curve denoted $C \subset \mathbb{C}$, such that there exists a parameter $\lambda \in [\alpha, \beta]$ with $\alpha,\beta \in \mathbb{R} \cup \{ -\infty, \infty \}$ such that $C$ is described as below:
$$\large s = \sigma + i\tau \in C, \sigma = f(\lambda), \tau = g(\lambda)$$
If $C$ is closed over $\mathbb{C}$, let there exist an integral $I_c$ defined as below:
$$\large I_c = \oint_C f(z)\ dz$$
If $C$ is not a closed contour over $\mathbb{C}$, let there exist an integral $I_o$ defined as below:
$$\large I_o = \int_C f(z)\ dz$$
We also define $f(z) = u(x + iy) + iv(x + iy)$ with $z = x + iy$ (assuming a closed contour $C$). Hence:
$$\large I_c = \oint_C u(z)\ dz + i \oint_C v(z)\ dz, I_o = \oint_C u(z)\ dz + i \oint_C v(z)\ dz$$
Let there exist a vector field denoted $\vec{v}(z)$ for $z \in \mathbb{C}$, such that $\vec{v}(z) = \overline{f(z)}$.
Note that $\mathbb{C}$ can be described as Euclidean space, under the angle definition:
$$\large e ^ {i\theta} = \cos(\theta) + i\ \sin(\theta)$$
Let these angles be directional. Define a set of 3 angles $(\alpha, \beta, \theta)$ with respect to an $s' \in C$ with $H \triangleq f(s'), \bar{H} \triangleq \overline{f(s')}, dz = dx + i\ dy = d\ \Re(z) + i\ d\ \Im(z)$ as such:
$$\large \alpha = \arctan\left(\frac{dy}{dx}\right), \beta = \arg(H), \theta = \alpha + \beta$$
Therefore, we obtain the below identities (let $\vec{T}$ and $\hat{n}$ be the tangent and normal vectors at $s'$ with respect to $C \in \mathbb{C}$):
$$\large \bar{H} = |H| e^{-i \beta}, \cos(\theta) = \left|\frac{\bar{H} \cdot \vec{T}}{\bar{H}}\right|, \sin(\theta) = \left|\frac{\bar{H} \cdot \hat{n}}{\bar{H}}\right|$$
Now, we evaluate for the integral (assuming a closed contour $C$) where $f \cong f(s)$ for valid $s \in \mathbb{C}$:
$$\large I_c = \oint_C f\ dz = \oint_C f\ (e^{i \alpha}\ dt)$$
$$\large f = |f|e^{i \beta} \Rightarrow I_c = \oint_C |f|e^{i \theta}\ dz$$
$$\large \oint_C |f|e^{i \theta}\ dz = \oint_C |\bar{f}| \cos(\theta)\ dz + i \oint_C |\bar{f}| \sin(\theta)\ dz$$
Now, let $\bar{f} \cong \vec{v}$, where the function $\bar{f}$ is characterised under the action of the vector field $\vec{v}$:
$$\large \therefore \oint_C f(z)\ dz = \oint_C \left(\vec{v} \cdot \vec{T} \right)\ dz + i \oint_C \left(\vec{v} \cdot \hat{n} \right)\ dz$$
Thus, the action of the Pólya vector field on a contour can determine the value of a complex line integral, as a real/imaginary composition of the work done by $\vec{v}$ on $C$ and the flux of $\vec{v}$ over $C$.
\begin{flushright} $\blacksquare$ \end{flushright}
\end{document}
